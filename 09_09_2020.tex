\documentclass{article}

\usepackage[english]{babel}
\usepackage{microtype}
\usepackage{graphicx}
\usepackage{wrapfig}
\usepackage{enumitem}
\usepackage{fancyhdr}
\usepackage{amsmath}
\usepackage{index}
\usepackage{hyperref}
\usepackage[margin=1.0in]{geometry}

\begin{document}
\title{Physics Notes}
\author{Dowland Aiello}
\date{September 09, 2020}

\maketitle
\tableofcontents
\fancyhf{}

\newpage

\section{Displacement}

\begin{itemize}
	\item \textbf{Displacement} can be described as an object's change in position relative to a reference frame,
		and can be expressed mathematically as follows:
		\[ \Delta x = x_f - x_0 \]
\end{itemize}

\subsection{Example}

A cyclist rides 3 km west and the turns around and rides 2 km east. (a) What is
her displacement? (b) What distance does she ride? (c) What is the magnitude of
her displacement?

\begin{enumerate}
	\item Her displacement is -1:
		\[ \Delta x = x_f - x_0; \Delta x = -1 - 0; \Delta x = -1 km \]
	\item She traveled 5 kilometers:
		\[ 3km + 2km = 5km \]
	\item The magnitude of her displacement is 1:
		\[ \lvert -1 \rvert = 1 \]
\end{enumerate}

\section{Vectors, Scalars, and Coordinate Systems}

\section{Time, Velocity, and Speed}

\begin{itemize}
	\item Time can be best defined in terms of how it is measured. Thus, the
		following definition would suffice in the context of physics: time is
		an invertal over which change occurs.
\end{itemize}

\section{Acceleration}

\section{Motion Equations for Constant Acceleration in One Dimension}

\section{Falling Objects}

\section{Graphical Analysis of One Dimensional Motion}

\end{document}
